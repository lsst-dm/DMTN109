\addtocounter{table}{-1}
\begin{longtable}{|p{0.145\textwidth}|p{0.8\textwidth}|}\hline
\textbf{Acronym} & \textbf{Description}  \\\hline

ADES & Astrometry Data Exchange Standard \\\hline
AURA & \gls{Association of Universities for Research in Astronomy} \\\hline
Alert & A packet of information for each source detected with signal-to-noise ratio > 5 in a difference image during \gls{Prompt Processing}, containing measurement and characterization parameters based on the past 12 months of \gls{LSST} observations plus small cutouts of the single-visit, template, and difference images, distributed via the internet \\\hline
Alert Production & The principal component of \gls{Prompt Processing} that processes and calibrates incoming images, performs \gls{Difference Image Analysis} to identify DIASources and DIAObjects, packages and distributes the resulting Alerts, and runs the \gls{Moving Object Processing System} \\\hline
Archive & The repository for documents required by the \gls{NSF} to be kept. These include documents related to design and development, construction, integration, test, and operations of the \gls{LSST} observatory system. The archive is maintained using the enterprise content management system \gls{DocuShare}, which is accessible through a link on the project website www.project.lsst.org \\\hline
Archive \gls{Center} & Part of the \gls{LSST} \gls{Data Management System}, the \gls{LSST} archive center is a data center at \gls{NCSA} that hosts the \gls{LSST} \gls{Archive}, which includes released science data and \gls{metadata}, observatory and engineering data, and supporting software such as the \gls{LSST} \gls{Software Stack} \\\hline
Association Pipeline & An application that matches detected Sources or DIASources or generated Objects to an existing catalog of Objects, producing a (possibly many-to-many) set of associations and a list of unassociated inputs. Association Pipelines are used in \gls{Prompt Processing} after \gls{DIASource} generation and in the final stages of \gls{Data Release} processing to ensure continuity of \gls{Object} identifiers \\\hline
Association of Universities for Research in Astronomy &  consortium of \gls{US} institutions and international affiliates that operates world-class astronomical observatories, \gls{AURA} is the legal entity responsible for managing what it calls independent operating Centers, including \gls{LSST}, under respective cooperative agreements with the \gls{National Science Foundation}. \gls{AURA} assumes fiducial responsibility for the funds provided through those cooperative agreements. \gls{AURA} also is the legal owner of the \gls{AURA} Observatory properties in Chile \\\hline
B & Byte (8 bit) \\\hline
CCD & \gls{Charge-Coupled Device} \\\hline
Camera & The \gls{LSST} subsystem responsible for the 3.2-gigapixel \gls{LSST} \gls{camera}, which will take more than 800 panoramic images of the sky every night. \gls{SLAC} leads a consortium of \gls{Department of Energy} laboratories to design and build the \gls{camera} sensors, optics, electronics, cryostat, filters and filter exchange mechanism, and \gls{camera} control system \\\hline
Center & An entity managed by \gls{AURA} that is responsible for execution of a federally funded project \\\hline
Charge-Coupled Device & a particular kind of solid-state sensor for detecting optical-band photons. It is composed of a 2-D array of pixels, and one or more read-out amplifiers \\\hline
Construction & The period during which \gls{LSST} observatory facilities, components, hardware, and software are built, tested, integrated, and commissioned. \gls{Construction} follows design and development and precedes operations. The \gls{LSST} construction phase is funded through the \gls{NSF} \gls{MREFC} account \\\hline
DCR & \gls{Differential Chromatic Refraction} \\\hline
DIA & \gls{Difference Image Analysis} \\\hline
DIAObject & A \gls{DIAObject} is the association of DIASources, by coordinate, that have been detected with signal-to-noise ratio greater than 5 in at least one difference image. It is distinguished from a regular Object in that its brightness varies in time, and from a SSObject in that it is stationary (non-moving) \\\hline
DIASource & A \gls{DIASource} is a detection with signal-to-noise ratio greater than 5 in a difference image \\\hline
DM & \gls{Data Management} \\\hline
DMS & \gls{Data Management} \gls{Subsystem} \\\hline
DMTN & \gls{DM} Technical Note \\\hline
DOE & \gls{Department of Energy} \\\hline
DR & \gls{Data Release} \\\hline
DRP & \gls{Data Release Production} \\\hline
Data Management & The \gls{LSST} Subsystem responsible for the \gls{Data Management System} (\gls{DMS}), which will capture, store, catalog, and serve the \gls{LSST} dataset to the scientific community and public. The DM team is responsible for the \gls{DMS} architecture, applications, middleware, infrastructure, algorithms, and Observatory Network Design. DM is a distributed team working at \gls{LSST} and partner institutions, with the DM \gls{Subsystem Manager} located at \gls{LSST} headquarters in Tucson \\\hline
Data Management System & The computing infrastructure, middleware, and applications that process, store, and enable information extraction from the \gls{LSST} dataset; the \gls{DMS} will process peta-scale data volume, convert raw images into a faithful representation of the universe, and archive the results in a useful form. The infrastructure layer consists of the computing, storage, networking hardware, and system software. The middleware layer handles distributed processing, data access, user interface, and system operations services. The applications layer includes the data pipelines and the science data archives' products and services \\\hline
Data Release & The approximately annual reprocessing of all \gls{LSST} data, and the installation of the resulting data products in the \gls{LSST} Data Access Centers, which marks the start of the two-year proprietary period \\\hline
Data Release Production & An episode of (re)processing all of the accumulated \gls{LSST} images, during which all output \gls{DR} data products are generated. These episodes are planned to occur annually during the \gls{LSST} survey, and the processing will be executed at the \gls{Archive} \gls{Center}. This includes Difference Imaging Analysis, generating deep Coadd Images, \gls{Source} detection and association, creating Object and \gls{Solar System Object} catalogs, and related \gls{metadata} \\\hline
Department of Energy & cabinet department of the United States federal government; the \gls{DOE} has assumed technical and financial responsibility for providing the \gls{LSST} \gls{camera}. The \gls{DOE}'s responsibilities are executed by a collaboration led by \gls{SLAC} National Accelerator Laboratory \\\hline
Difference Image & Refers to the result formed from the pixel-by-pixel difference of two images of the sky, after warping to the same pixel grid, scaling to the same photometric response, matching to the same \gls{PSF} \gls{shape}, and applying a correction for \gls{Differential Chromatic Refraction}. The pixels in a difference thus formed should be zero (apart from noise) except for sources that are new, or have changed in brightness or position. In the \gls{LSST} context, the difference is generally taken between a visit image and template.  \\\hline
Difference Image Analysis & The detection and characterization of sources in the \gls{Difference Image} that are above a configurable threshold, done as part of \gls{Alert} Generation Pipeline \\\hline
Differential Chromatic Refraction & The refraction of incident light by Earth's atmosphere causes the apparent position of objects to be shifted, and the size of this shift depends on both the wavelength of the source and its \gls{airmass} at the time of observation. \gls{DCR} corrections are done as a part of \gls{DIA} \\\hline
DocuShare & The trade name for the enterprise management software used by \gls{LSST} to archive and manage documents \\\hline
ESA & European Space Agency \\\hline
FITS & \gls{Flexible Image Transport System} \\\hline
Flexible Image Transport System & an international standard in astronomy for storing images, tables, and \gls{metadata} in disk files. See the \gls{IAU} \gls{FITS} Standard for details \\\hline
GB & Gigabyte \\\hline
IAU & International Astronomical Union \\\hline
IVOA & International Virtual-Observatory Alliance \\\hline
JPL & Jet Propulsion Laboratory (DE ephemerides) \\\hline
LSST & Large Synoptic Survey Telescope \\\hline
MB & MegaByte \\\hline
MBA & Main Belt Asteroid \\\hline
MOC & Multi Ordered Catalogue \\\hline
MOPS & \gls{Moving Object Processing System} \\\hline
MPC & \gls{IAU} Minor Planet \gls{Center} \\\hline
MREFC & Major Research Equipment and Facility \gls{Construction} \\\hline
Major Research Equipment and Facility \gls{Construction} & the \gls{NSF} account through which large facilities construction projects such as \gls{LSST} are funded \\\hline
NASA & National Aeronautics and Space Administration \\\hline
NCSA & National \gls{Center} for Supercomputing Applications \\\hline
NDO & non-discoverable object \\\hline
NEO & Near-Earth \gls{Object} \\\hline
NSF & \gls{National Science Foundation} \\\hline
National Science Foundation & primary federal agency supporting research in all fields of fundamental science and engineering; \gls{NSF} selects and funds projects through competitive, merit-based review \\\hline
Object & In \gls{LSST} nomenclature this refers to an \gls{astronomical object}, such as a star, galaxy, or other physical entity. E.g., comets, asteroids are also Objects but typically called a Moving Object or a \gls{Solar System Object} (SSObject). One of the \gls{DRP} data products is a table of Objects detected by \gls{LSST} which can be static, or change brightness or position with time \\\hline
Operations & The 10-year period following construction and commissioning during which the \gls{LSST} Observatory conducts its survey \\\hline
PHA & Potentially Hazardous Asteroid \\\hline
PSF & Point Spread Function \\\hline
Pan-STARRS & Panoramic Survey Telescope and Rapid Response System \\\hline
Project Manager & The person responsible for exercising leadership and oversight over the entire \gls{LSST} project; he or she controls schedule, budget, and all contingency funds \\\hline
Prompt Processing & The processing that occurs at the \gls{Archive} \gls{Center} on the nightly stream of raw images coming from the telescope, including Difference Imaging Analysis, \gls{Alert} Production, and the \gls{Moving Object Processing System}. This processing generates Prompt Data Products \\\hline
SED & \gls{Spectral Energy Distribution} \\\hline
SI & Syst\`eme International (International System of units defined by ISO) \\\hline
SLAC & \gls{SLAC} National Accelerator Laboratory (formerly Stanford Linear Accelereator \gls{Center}; \gls{SLAC} is now no longer an acronym) \\\hline
SSA & Space Situational Awareness \\\hline
SSO & \gls{Solar System Object} \\\hline
SSP & Solar System Processing \\\hline
Science Pipelines & The library of software components and the algorithms and processing pipelines assembled from them that are being developed by \gls{DM} to generate science-ready data products from \gls{LSST} images. The Pipelines may be executed at scale as part of \gls{LSST} Prompt or \gls{Data Release} processing, or pieces of them may be used in a standalone mode or executed through the \gls{LSST} \gls{Science Platform}. The \gls{Science Pipelines} are one component of the \gls{LSST} \gls{Software Stack} \\\hline
Science Platform & A set of integrated web applications and services deployed at the \gls{LSST} Data Access Centers (DACs) through which the scientific community will access, visualize, and perform next-to-the-data analysis of the \gls{LSST} data products \\\hline
Software Stack & Often referred to as the \gls{LSST} Stack, or just The Stack, it is the collection of software written by the \gls{LSST} \gls{Data Management} Team to process, generate, and serve \gls{LSST} images, \gls{transient} alerts, and catalogs. The Stack includes the \gls{LSST} \gls{Science Pipelines}, as well as packages upon which the \gls{DM} software depends. It is open source and publicly available \\\hline
Solar System \gls{Object} & A solar system object is an astrophysical object that is identified as part of the Solar System: planets and their satellites, asteroids, comets, etc. This class of object had historically been referred to within the \gls{LSST} Project as Moving Objects \\\hline
Source & A single detection of an astrophysical object in an image, the characteristics for which are stored in the \gls{Source} Catalog of the \gls{DRP} database. The association of Sources that are non-moving lead to Objects; the association of moving Sources leads to Solar System Objects. (Note that in non-LSST usage "source" is often used for what \gls{LSST} calls an \gls{Object}.) \\\hline
Spectral Energy Distribution & the radiated energy of an astrophysical object as a function of energy (or wavelength) across the entire spectrum of light \\\hline
Subsystem & A set of elements comprising a system within the larger \gls{LSST} system that is responsible for a key technical deliverable of the project \\\hline
Subsystem Manager & responsible manager for an LSST subsystem; he or she exercises authority, within prescribed limits and under scrutiny of the Project Manager, over the relevant subsystem's cost, schedule, and work plans \\\hline
TNO & Trans-Neptunian \gls{Object} \\\hline
UNO & Unassociated object \\\hline
US & United States \\\hline
Validation & A process of confirming that the delivered system will provide its desired functionality; overall, a validation process includes the evaluation, integration, and test activities carried out at the system level to ensure that the final developed system satisfies the intent and performance of that system in operations \\\hline
Wide-Fast-Deep & The main survey of the \gls{LSST} to cover at least 18000 square degrees of the southern sky \\\hline
XMM & X-ray Multi-mirror Mission (\gls{ESA}; officially known as \gls{XMM}-Newton) \\\hline
airmass & The pathlength of light from an astrophysical source through the Earth's atmosphere. It is given approximately by sec z, where z is the angular distance from the zenith (the point directly overhead, where \gls{airmass} = 1.0) to the source \\\hline
astronomical object & A star, galaxy, asteroid, or other physical object of astronomical interest. Beware: in non-LSST usage, these are often known as sources \\\hline
camera & An imaging device mounted at a telescope focal plane, composed of optics, a shutter, a set of filters, and one or more sensors arranged in a focal plane array \\\hline
declination & Often abbreviated Dec, it is a part of an equatorial coordinate pair that expresses the angular distance (usually expressed in degrees) from the Celestial Equator, measured along great circles that intersect the Equatorial poles. Positions south of the equator are given negative sign \\\hline
epoch & Sky coordinate reference frame, e.g., J2000. Alternatively refers to a single observation (usually photometric, can be multi-band) of a variable source \\\hline
flux & Shorthand for radiative \gls{flux}, it is a measure of the transport of radiant energy per unit area per unit time. In astronomy this is usually expressed in cgs units: erg/cm2/s \\\hline
footprint & See 'source footprint', 'instrumental footprint', or 'survey footprint', `Footprint` is a Python class representing a \gls{source footprint} \\\hline
forced photometry & A measurement of the photometric properties of a source, or expected source, with one or more parameters held fixed. Most often this means fixing the location of the center of the brightness profile (which may be known or predicted in advance), and measuring other properties such as total brightness, \gls{shape}, and orientation. Forced photometry will be done for all Objects in the \gls{Data Release Production} \\\hline
metadata & General term for data about data, e.g., attributes of astronomical objects (e.g. images, sources, astroObjects, etc.) that are characteristics of the objects themselves, and facilitate the organization, preservation, and query of data sets. (E.g., a \gls{FITS} header contains \gls{metadata}) \\\hline
pipeline & A configured sequence of software tasks (Stages) to process data and generate data products. Example: \gls{Association Pipeline} \\\hline
precovery & The process of finding, or putting upper limits on, detections of a newly discovered \gls{DIAObject} in previously obtained images, typically using \gls{forced photometry}. \gls{Alert} Packets will contain \gls{precovery} data derived from the past 30 days of images that include the location of a new \gls{DIAObject} \\\hline
shape & In reference to a \gls{Source} or Object, the \gls{shape} is a functional characterization of its spatial intensity distribution, and the integral of the \gls{shape} is the \gls{flux}. Shape characterizations are a data product in the \gls{DIASource}, \gls{DIAObject}, \gls{Source}, and Object catalogs \\\hline
transient & A \gls{transient} source is one that has been detected on a difference image, but has not been associated with either an \gls{astronomical object} or a solar system body \\\hline
\end{longtable}
